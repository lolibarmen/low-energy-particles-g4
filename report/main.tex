\documentclass[a4paper,12pt]{article} %размер бумаги устанавливаем А4, шрифт 12пунктов
\usepackage[T2A]{fontenc}
\usepackage[utf8]{inputenc} %кодировка
\usepackage[english,russian]{babel}%используем русский и английский языки с переносами
\usepackage{amssymb,amsfonts,amsmath,cite,enumerate,float,indentfirst,titlesec} %пакеты расширений
\usepackage{graphicx} %вставка графики
\usepackage{listings}


\makeatletter
\renewcommand{\@biblabel}[1]{#1.} % Заменяем библиографию с квадратных скобок на точку:
\makeatother

\usepackage{geometry} % Меняем поля страницы
\geometry{left=2cm}% левое поле
\geometry{right=1.5cm}% правое поле
\geometry{top=1cm}% верхнее поле
\geometry{bottom=2cm}% нижнее поле

\newcommand{\imgh}[3]{\begin{figure}[h]\center{\includegraphics[width=#1]{#2}}\caption{#3}\label{ris:#2}\end{figure}} % Для вставки картинок
% \newcommand{\svgh}[3]{\begin{figure}[h]\center{\includesvg[width=#1]{#2}}\caption{#3}\label{ris:#2}\end{figure}\newline} % Для вставки картинок

\makeatletter
\renewcommand{\section}{%
  \@startsection{section}{1}{\z@}%
  {-3.5ex \@plus -1ex \@minus -.2ex}%
  {2.3ex \@plus.2ex}%
  {\normalfont\Large\bfseries}%
}
\renewcommand{\@seccntformat}[1]{\csname the#1\endcsname.\quad}
\makeatother

\begin{document}

\begin{titlepage}
\newpage
\begin{center}
ФГАОУ ВО «Уральский федеральный университет имени \\
первого Президента России Б. Н. Ельцина» \\ Физико-технологический институт \\ Кафедра теоретической физики и прикладной математики
\end{center}
\vspace{16em}
\begin{center}
\Large \textbf{Отчет по итерации индивидуального проекта. \\
Итерация №1 <<???>>}
\end{center}
\vspace{16em}
\hfill\parbox{7cm}{ % тут регулировать расстояние "Студент/Преподаватель/Группа" до правого края
Студент:\hfill Агапов Д.П.\\
Преподаватель:\hfill Вазиров Р.А.\\
Группа:\hfill ФТ-330005\\
}
\vspace{\fill}
\begin{center}
Екатеринбург \\2025
\end{center}
\end{titlepage}

\newpage
\section{Цель работы}
Практическое использование библиотеки geant4 на примере системы "Излучатель электронов(1MeV, 10MeV) + водная мишень". Проверить совпадение построенной модели с теоретическими и практическими данными.
\section{Геометрия модели}
Излучатель электронов с настраиваемой энергией $E_{beam}$, положением $\vec r_{gun}$ и направлением вектора нормали $\vec n_{gun}$ к прямоугольной области $\Omega_{bean}$. Прямоугольная область строится так: строится плоскость, задаваемая вектором нормали $\vec n_{gun}$ и точкой положения пушки $\vec r_{gun}$. Далее строиться прямоугольная область $\Omega_{bean}$ с фиксированной стороной $10cm$ так, чтобы точка $\vec r_{gun}$ являлась точкой пересечения диагоналей прямоугольной области(центром прямоугольной области).

Водный фантом(мишень) имеющий форму параллелограмма со сторонами\footnote{Считается, что стороны направленны вдоль осей, индексы означают ось, вдоль которой направлена сторона} $l_x = 30cm$, $l_y=30cm$, $l_z=20cm$ и центром совпадающим с центром координат. Объём вне фантома заполнен воздухом. Воздух в Geant4 считается как газ с плотностью $1.29 * 10^{-3}g/cm^3$, являющийся смесью следующих газов: Азот - 70\%, кислород - 23\%, аргон - 1\%. 
\section{Динамика модели}

Излучатель испускает электроны с заданной энергией $E_{\text{beam}}$ из случайной точки прямоугольной области $\Omega_{\text{beam}}$, направляя частицы параллельно вектору $\vec n_{\text{gun}}$. Частицы транспортируются через среду фантома(вода) с полной симуляцией стахастических взаимодействий, которые описываются набором физических процессов Geant4.

При прохождении через фантом отдельная частица на каждом шаге может:
\begin{itemize}
  \item испытывать множественное упругое и неупругое рассеяние (изменение направления движения и передача энергии среде);
  \item терять энергию через ионизацию и возбуждение молекул среды;
  \item порождать вторичные электроны и фото/комптон/брекет-фотоны;
  \item испускать тормозное излучение при взаимодействии с полем атомных ядер;
  \item вызывать атомную деэкситацию — флуоресценцию и выброс электронов Оже при захвате вакансий;
  \item при наличии высокоэнергетических фотонов — возможны фотоэффект, комптон-рассеяние и, при достижении порога, процессы, связанные с образованием пар и фотоядерные реакции (вклад маловероятен при энергиях ближе к 1 MeV, возрастает при более высоких энергиях и/или при наличии высокоэнергетических фотонов).
  \item и т.д.
\end{itemize}

Вероятности и конкретный набор процессов в симуляции задаются зарегистрированными модулями физики Geant4 (см. раздел «Список физики»). Для точного моделирования на малых энергиях включены специализированные низкоэнергетические модели и атомные эффекты, что важно для корректного расчёта локальной дозы и образования вторичных электронов.

\section{Список физики}

Для моделирования взаимодействия электронов с веществом используется составная физическая модель, основанная на стандартных и низкоэнергетических пакетах Geant4, оптимизированных для медицинской и дозиметрической физики. Ниже приведено описание подключённых модулей.

\begin{itemize}
  \item \textbf{G4DecayPhysics} — реализует распад нестабильных элементарных частиц и ядер. В данной задаче основная роль данного модуля заключается в обеспечении корректного поведения возможных короткоживущих вторичных частиц.

  \item \textbf{G4RadioactiveDecayPhysics} — описывает процессы радиоактивного распада нестабильных нуклидов и связанные с ними продукты (альфа-, бета-, гамма-излучение). Хотя для электронного пучка и водного фантома вклад таких процессов невелик, модуль обеспечивает физическую полноту модели.

  \item \textbf{G4EmExtraPhysics} — включает дополнительные электромагнитные процессы, такие как генерация гамма-квантов при аннигиляции позитронов, фотоядерные реакции и другие эффекты, дополняющие стандартный набор.

  \item \textbf{G4HadronElasticPhysics} — реализует упругое рассеяние адронов на ядрах. Актуален при возможном образовании вторичных адронов в результате высокоэнергетических фотонных или ядерных взаимодействий.

  \item \textbf{G4HadronPhysicsFTFP\_BERT} — описывает неупругие ядерные взаимодействия адронов с ядрами вещества, включая каскад Берта и модель Фрей-Шенкенберга (FTFP). Вклад этих процессов для электронных пучков мал, но модуль активен для корректной обработки редких вторичных адронных событий.

  \item \textbf{G4StoppingPhysics} — моделирует процессы торможения и захвата тяжёлых заряженных частиц в веществе. Для электронов данный модуль обеспечивает корректное завершение трека при полном поглощении энергии.

  \item \textbf{G4IonPhysics} — реализует взаимодействие ионов с веществом, включая ионизацию, возбуждение и ядерные процессы. Используется для моделирования вторичных ионов, которые могут возникать при взаимодействиях высокоэнергетических частиц.

  \item \textbf{G4EmStandardPhysics\_option4} — стандартный набор электромагнитных процессов с расширенной точностью, оптимизированный для медицинской физики. Включает процессы ионизации, возбуждения, тормозного излучения, множественного рассеяния, взаимодействий фотонов (фотоэффект, комптон-рассеяние, парное рождение) и аннигиляции позитронов. Данная опция обеспечивает высокую точность расчёта пробега и дозы при энергиях в диапазоне низких энергий.

  \item \textbf{G4EmLowEPPhysics} — модуль низкоэнергетической электромагнитной физики, использующий детализированные модели взаимодействия электронов, фотонов и ионов при энергиях вплоть до нескольких эВ. Учитывает атомную деэкситацию (флуоресценцию и эмиссию электронов Оже), что позволяет точно моделировать распределение дозы на микроуровне.

  \item \textbf{Настройка электромагнитных параметров (ConfigureEMPhysics)} — определяет пороги генерации вторичных частиц (production cuts), шаг интегрирования и другие параметры транспортировки. Правильная настройка этих параметров обеспечивает баланс между точностью моделирования и скоростью расчёта.
\end{itemize}

Совокупность перечисленных модулей обеспечивает полное описание электромагнитных взаимодействий электронов с веществом в диапазоне энергий 1–10~МэВ, а также корректное моделирование побочных фотонных и (при необходимости) ядерных процессов. Такой набор соответствует рекомендациям Geant4 для медицинских и дозиметрических расчётов и обеспечивает реалистичное распределение дозы в водном фантоме.

Использование энергии

\end{document}